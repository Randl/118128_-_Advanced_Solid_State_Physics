\documentclass[]{article}
\usepackage{amsmath}
\usepackage{amsfonts}
\usepackage{amssymb}
\usepackage{gensymb}
\usepackage{graphicx}
\usepackage{svg}
\usepackage{bbding}
\usepackage{mathtools}
\usepackage{centernot} % not parallel, etc.
\usepackage{lmodern}
\usepackage{morewrites}
\usepackage{xcolor,sectsty} % colorful sections
\usepackage[left=10mm, top=10mm, right=10mm, bottom=20mm, nohead]{geometry}
%\usepackage{bigints}
\usepackage{dsfont} %mathbb 1
\usepackage{esint} % beatiful integrals
\usepackage{physics}
\usepackage{amsthm} % theorems
\usepackage{natbib}


\usepackage{varioref} % references to show page
\usepackage[pagebackref]{hyperref}
\usepackage{url}
\usepackage[capitalise]{cleveref} 

\newcommand{\yrcite}[1]{\citeyearpar{#1}}
\renewcommand{\cite}[1]{\citep{#1}}

\usepackage[T1]{fontenc}
% Nicer default font (+ math font) than Computer Modern for most use cases
% \usepackage{mathpazo} % problems with greek vectors
\usepackage[utf8x]{inputenc} % Allow utf-8 characters in the tex document
% Prevent overflowing lines due to hard-to-break entities
\sloppy 
% Colors for the hyperref package
\definecolor{urlcolor}{rgb}{0,.145,.698}
\definecolor{linkcolor}{rgb}{.71,0.21,0.01}
\definecolor{citecolor}{rgb}{.12,.54,.11}
% Setup hyperref package
\hypersetup{
	breaklinks=true,  % so long urls are correctly broken across lines
	colorlinks=true,
	urlcolor=urlcolor,
	linkcolor=linkcolor,
	citecolor=citecolor,
}


\DeclareFontFamily{OMX}{lmex}{}
\DeclareFontShape{OMX}{lmex}{m}{n}{<-> lmex10}{}


%colors of sections
\definecolor{secfont}{RGB}{46,116,181}
\definecolor{subfont}{RGB}{146,23,57}
\definecolor{parfont}{RGB}{19,127,43}
\definecolor{subparfont}{RGB}{7,11,100}

\subsectionfont{\color{subfont}}
\sectionfont{\color{secfont}}
\paragraphfont{\color{parfont}}
\subparagraphfont{\color{subparfont}}


% declare a new theorem style
\newtheoremstyle{bluestyle}%
{3pt}% Space above
{3pt}% Space below 
{}% Body font
{}% Indent amount
{\bfseries\color{blue}}% Theorem head font
{.}% Punctuation after theorem head
{.5em}% Space after theorem head
{}% Theorem head spec (can be left empty, meaning ‘normal’)
% declare a new theorem style
\newtheoremstyle{redstyle}{3pt}{3pt}{}{}{\bfseries\color{red}}{.}{.5em}{}
\newtheoremstyle{olivestyle}{3pt}{3pt}{}{}{\bfseries\color{olive}}{.}{.5em}{}
\newtheoremstyle{orangestyle}{3pt}{3pt}{}{}{\bfseries\color{orange}}{.}{.5em}{}
\newtheoremstyle{magentastyle}{3pt}{3pt}{}{}{\bfseries\color{magenta}}{.}{.5em}{}

\theoremstyle{bluestyle}
\newtheorem{theorem}{Theorem}[section]
\theoremstyle{redstyle}
\newtheorem{definition}{Definition}[section]
\theoremstyle{magentastyle}
\newtheorem{coll}{Collary}[theorem]
\theoremstyle{olivestyle}
\newtheorem{lemma}{Lemma}[section]
\theoremstyle{olivestyle}
\newtheorem{prop}[theorem]{Proposition}

%\usepackage{babel}[english]
%opening
\title{118028 -- Quantum Transport in Solids}
\author{Assa Auerbach}


\parindent=0em
\begin{document}


\maketitle

\begin{abstract}

\end{abstract}

%\tableofcontents
\section{Introduction}
The simplest Hamiltonian is, denoting electrons with $i$ and ions with $I$: 
\begin{align}
H = \sum_i \frac{\vb{P}_i^2}{2m_e} +  \sum_I \frac{\vb{P}_I^2}{2m_I} +  \sum_{i,I} \frac{e^2Z}{\abs{\vb{r}_i - \vb{R}_i}} +  \sum_{i,j} \frac{e^2}{\abs{\vb{r}_i - \vb{r}_j}}+  \sum_{I,J} \frac{e^2Z^2}{\abs{\vb{R}_i - \vb{R}_j}}
\end{align}
The last part of Hamiltonian is many-body interaction, which makes it complex, due to big amount of degrees of freedom. 

We neglect here relativistic (spin-orbit) and radioactive corrections. We'll add electromagnetic field interaction later.

We note that $\frac{m_e}{m_I} \lesssim 10^3$, and perform Born-Oppenheimer approximation: neglecting ion movement (frozen ions). Yet electron-electron interactions is many-body problem:

\begin{align}
H^{el} = \sum_i \frac{\vb{P}_i^2}{2m_e} +  \sum_i V\qty(\vb{r}_i) + \sum_{i,j} \frac{e^2}{\abs{\vb{r}_i - \vb{r}_j}}
\end{align}

The problem is still pretty complex one. If we have 200 one-particle states and we put 100 particles in the system. In case of fermions there are $\binom{200}{100}$ possible states and and for bosons $\binom{200+100-1}{100}$.

There are phenomena that happen only in large scale: spontaneously broken symmetry, critical phenomena, scale invariance.

\paragraph{Organizing principle}
\begin{enumerate}
	\item Symmetry: $\mathbb{Z}_2, O(2), O(3)$
	\item Statistics of particles (Fermi or Bose statistics)
	\item Range of interactions.
	\item Gauge fields. 
	\item Thermal and quantum fluctuations.
	\item Topological invariants.
\end{enumerate}

\paragraph{Particles in the box}
\begin{align}
H = \sum_i \frac{\vb{P}_i^2}{2m_e}
\end{align}

We use periodic boundary conditions (PBC) and then single-particle eigenfucntions are plane waves:
\begin{align}
\epsilon_{\vb{k}} &= \frac{\hbar \vb{k}^2}{2m}\\
\braket{\vb{x}}{\vb{k}} &= \frac{1}{\sqrt{V}} e^{i \vb{k} \vdot \vb{x}}
\end{align}
Wavenumbers are thus quantized as $k_i = \frac{2\pi}{L_i}n_i$

\paragraph{Density of states}
We define density of states:
\begin{align}
\mathcal{N}(\epsilon) = \frac{1}{V}\sum_{\vb{k}} \tilde{\delta}\qty(\epsilon - \epsilon_{\vb{k}})
\end{align}
$\tilde{\delta}$ approximates delta function with some function wider than distance between two allowed energies.
Switching to integral:

\begin{align}
\mathcal{N}^{3D}(\epsilon) = 2\cdot \int \frac{\dd[3]{k}}{(2\pi)^3} \delta(\epsilon - \epsilon_{\vb{k}}) &= \frac{m}{\hbar^3 \pi^2} \sqrt{2m \epsilon}\\
\mathcal{N}^{2D}(\epsilon) &= \frac{m}{n}\\
\mathcal{N}^{1D}(\epsilon) &= \frac{1}{\sqrt{\epsilon}}
\end{align}

\begin{theorem}[Bloch theorem] \label{bloch}
	Eigenfunction of periodic Hamiltonian are plane waves times periodic function.
\begin{proof}
	Take a look on a single particle Hamiltonian:
	\begin{align}
	H = \frac{\vb{p}^2}{2m} + V^{eff} (\vb{r})
	\end{align}
	If $V^{eff}(\vb{r}) = V^{eff}(\vb{R} + \vb{R}_n)$ for $\vb{R}_n$ lattice vector, then Hamiltonian is symmetric under translations:
	\begin{align}
	T^\dagger(\vb{R}_n) H T(\vb{R}_n) = H
	\end{align}
	$T$ defines Abelian group, and thus $T$ is unitary operator. Since $T$ commutes with $H$, we can diagonalize $H$ and $T$ simultaneously:
	\begin{align}
	T(\vb{R}_n) \ket{\psi_\alpha} &= e^{i \phi(\vb{R}_n, \alpha)} \ket{\psi_\alpha}\\
	H \ket{\psi_\alpha} &= \epsilon_\alpha \ket{\psi_\alpha}\
	\end{align}
	By looking on two consequent translations, we find out $\phi$ is linear:
	\begin{align}
	T(\vb{R}_1)T(\vb{R}_2) \ket{\psi_\alpha} &= e^{i \phi_\alpha(\vb{R}_1)}e^{i \phi_\alpha(\vb{R}_2)} \ket{\psi_\alpha}\\
	\phi_\alpha(\vb{R}_1 + \vb{R}_2) &=\phi_\alpha(\vb{R}_1) + \phi_\alpha(\vb{R}_2)
	\end{align}
	Thus 
	\begin{align}
	\phi_\alpha(\vb{R}) = \vb{K}_\alpha \vdot \vb{R}
	\end{align}
	
	From PBC we get
	\begin{align}
	K_i \cdot L_i = 2\pi n_i 
	\end{align}
	Since eigenfunctions of Hamiltonian are eigenfunctions of $T$, we conclude that 
	\begin{align}
	\psi_k(\vb{r}) = e^{i \vb{k} \vdot \vb{r}} \mathcal{U}(\vb{r})
	\end{align}
	for $\mathcal{U}(\vb{r}) = \mathcal{U}(\vb{r}+\vb{R})$.
\end{proof}
\end{theorem}

$\vb{k}$ are limited in their value by first Brillouin zone. We can rewrite Hamiltonian:
\begin{align}
H = \sum_{\vb{k}, \alpha} \epsilon_{\vb{k}, \alpha} \dyad{\psi_k^\alpha}
\end{align}

If $V$ has additional symmetries, for example, reflection symmetry, energies have same symmetries:
\begin{align}
R^{-1} V(\vb{R}) R = V(\vb{r}) \Rightarrow \epsilon_{\vb{k}} =  \epsilon_{R(\vb{k})}
\end{align}

While $\mathcal{U}$ determines short-term behavior of particles, $e^{i \vb{k} \vdot \vb{r}}$ governs long-term behavior.

\section{Tight binding model}
We are looking on bands of energies in a single Brillouin zone, especially interested in conductance band (one intersected by Fermi energy).

In tight-binding we rewrite Hamiltonian as nearest neighbor model:
 \begin{align}
 H = -\frac{t}{2} \sum_{\vb{R}} \dyad{\vb{R}}{\vb{R}+\vb{\eta}} + h.c.
 \end{align}
, where $\ket{\vb{R}}$ are local states such that $\braket{\vb{R}}{\vb{R}'} = \delta_{\vb{R},\vb{R}'}$.

\paragraph{Lattice Fourier transform}
In 1D, $\vb{R}_j = \vb{a}\cdot j$. Define 
\begin{align}
\vb{k} = \underbrace{\frac{1}{\sqrt{N}} \sum_j e^{ikaj}}_{U_{\vb{k}}} \ket{\vb{R}_j}
\end{align}
$U_{\vb{k}}$ is unitary matrix:
\begin{align}
\mel{j}{U^\dagger_{\vb{k}}U_{\vb{k}}}{j'} = \frac{1}{N} \sum_{j,j'} e^{-ikaj}e^{ikaj'} = \frac{1}{N} \sum_{j,j'} e^{-ika(j-j')} = \delta_{jj'}
\end{align}

Now
\begin{align}
\epsilon_{\vb{k}} =\mel{\vb{k}}{H}{\vb{k}} = -\frac{t}{N} \sum_{\vb{R}'} \sum_{\vb{R}, \vb{\eta}} e^{i \vb{k} \vdot \vb{R} } \bra{\vb{R}'}\dyad{\vb{R}}{\vb{R}+\vb{\eta}} e^{-i \vb{k} \vdot \vb{R}} = -\frac{1}{N} \sum_{\vb{R}, \vb{\eta}} e^{i\vb{k} \vdot \qty(\vb{R} - \vb{R} + \vb{\eta})} = -t \underbrace{\sum_{\vb{\eta}} e^{i \vb{k} \vdot \vb{\eta}} }_{\gamma_{\vb{k}}^{(\eta)}}
\end{align}

\paragraph{Examples}
\subparagraph{1D}
\begin{align}
\epsilon_{\vb{k}} = -2t \cos(ka)
\end{align}
\subparagraph{2D}
\begin{align}
\epsilon_{\vb{k}} = -2t \qty\big[\cos(k_x a)+\cos(k_y a)]
\end{align}

\subsection{Wannier States}
\paragraph{Reminder}
For lattice vectors $\vb{r}_i$, reciprocal lattice vectors are $\vb{k}_i = \frac{2\pi}{V} \cdot \qty(\vb{r}_j \cross \vb{r}_k)$.

\begin{definition}[Wannier states]
	Wannier states are lattice Fourier transform of Bloch wave.
	\begin{align}
	W^\alpha_n(\vb{r}) = \frac{1}{N} \sum_{\vb{k}} e^{-i \vb{k} \vdot \vb{R}_n} \psi_{\vb{k}} (\vb{r}) = \frac{1}{N} \sum_{\vb{k}} e^{-i \vb{k} \vdot \vb{R}_n} e^{i\vb{k} \vdot (\vb{r}-\vb{R}_n)}\mathcal{U}_{\vb{k}} (\vb{r}-\vb{R}_n) = W^\alpha (\vb{r}-\vb{R}_n)
	\end{align}
\end{definition}

\begin{prop}
	$W^\alpha_n$ are orthogonal.
	\begin{proof}
		\begin{align}
		\int \dd{\vb{r}} W^\alpha_n(\vb{r}) W^\beta_{n'}(\vb{r}) = \frac{1}{N^2} \sum_{\vb{k}, \vb{k}'} e^{i\qty(\vb{k} \vdot \vb{R}_n - \vb{k}' \vdot \vb{R}_{n'})} \int \dd{\vb{r}} {\psi^{\alpha}_{\vb{k}}}^* (\vb{r}) \psi^{\beta}_{\vb{k}'} (\vb{r}) = \frac{1}{N} \sum_{\vb{k}} e^{i \vb{k}(\vb{R}_n - \vb{R}_{n'})} =\delta_{nn'} \delta_{\alpha \beta}
		\end{align}
	\end{proof}
\end{prop}

How local are $W^\alpha$?
\begin{align}
W(\vb{r}) = \int \frac{\dd{\vb{k}}}{(2\pi)^3} e^{i\vb{k} \vdot \vb{r}} \mathcal{U}_{\vb{k}}(\vb{r})
\end{align}
Looking for a maximum of $\mathcal{U}_{\vb{k}}(\vb{r})$, we apply logarithm $f(\vb{r}) = \log(\mathcal{U}_{\vb{k}}(\vb{r}))$ and Taylor expand it:

\begin{align}
W(\vb{r}) = \int \frac{\dd{\vb{k}}}{(2\pi)^3} e^{i\vb{k} \vdot \vb{r}} \mathcal{U}_{\vb{k}_0}(\vb{r}_0) \exp(-\frac{1}{2} \pdv{\vb{k}} f_{\vb{k}_0} (\vb{r}_0) (\vb{k} - \vb{k}_0)^2 -\frac{1}{2} \pdv{\vb{r}} f_{\vb{k}_0} (\vb{r}_0) (\vb{r} - \vb{r}_0)^2)
\end{align}
Denote $\pdv{\vb{k}} f_{\vb{k}_0}  = a^2$, taking $R_n$ large enough and thus neglecting $\vb{r}$ derivative since $(\vb{r} - \vb{r}_0)^2$ is small we get:

\begin{align}
W(\vb{R}_n-\vb{r}_0) = \int \frac{\dd{\vb{k}}}{(2\pi)^3} e^{i\vb{k} \vdot \vb{R}_n} e^{-\frac{1}{2} a^2 (\vb{r}_0) (\vb{k} - \vb{k}_0)^2 } \sim e^{- \frac{\vb{R}_n^2}{2a^2}}
 \end{align}
 
 Thus
 \begin{align}
 W(\vb{r}-\vb{R}_n)  \sim e^{- \frac{(\vb{r}-\vb{R}_n)^2}{a^2}}
 \end{align}
 
If $\mathcal{U}_{\vb{k}}(\vb{r}_0)$ varies slowly with $\vb{k}$, Wannier states are localized. In an extreme case of 
$\mathcal{U}_{\vb{k}}^\alpha = \phi^\alpha(\vb{r})$:

\begin{align}
W_\alpha (\vb{r}- \vb{R}_n) = \int \dd{k} e^{i \vb{k}(\vb{r}-\vb{R}_n)} \phi_\alpha(\vb{r}-\vb{R}_n) = \delta(\vb{r} -\vb{R}_n)\phi_\alpha(\vb{r} -\vb{R}_n)
\end{align}

If $\mathcal{U}_{\vb{k}}$ is singular function of $\vb{k}$ we get
\begin{align}
W(\vb{r}-\vb{R}_n) \sim \frac{1}{(\vb{r}-\vb{R}_n)^\gamma}
\end{align}

\subsection{Tight-binding}
Tight-binding is good when there is a single band and Wannier function is slowly varying (at a momentum scale of Bruilien zone). 
We can write the Hamiltonian as
\begin{align}
H  = \frac{P^2}{2m_e} + V_{eff}(\vb{r}) =  \frac{P^2}{2m_e} + \sum_n V_{atom}(\vb{r}-\vb{R}_n) + \Delta V(\vb{r} - \vb{R}_n)
\end{align}
where $\Delta V$ is difference between atomic and lattice potentials. If wavefunctions are localized around nuclei the effect of $\Delta V$ is small.

We'll use anzatz of sum of local states:
\begin{align}
\tilde{\psi}_\alpha (\vb{r}) = \sum_{\vb{R}_n} e^{-i \vb{k} \vdot \vb{R}_n}\phi^\alpha (\vb{r}-\vb{R}_n)
\end{align}

Those wavefunctions are not eigenstates, but approximate them. We get
\begin{align}
\braket{\tilde{\psi}}{\tilde{\psi}'} = \int \dd{\vb{r}} \sum_{\vb{R}_n, \vb{R}_{n'}} e^{i \vb{k}\vdot \vb{R}_n}e^{-i \vb{k}'\vdot \vb{R}_{n'}}  {\phi}^* (\vb{r}-\vb{R}_n) \phi (\vb{r}-\vb{R}_n') 
	\end{align}
Defining $\alpha(\vb{R}_n - \vb{R}_{n'}) = \int \dd{\vb{r}} \ {\phi^\alpha}^* (\vb{r}-\vb{R}_n) \phi^\alpha (\vb{r}-\vb{R}_n') $ we get
\begin{align}
\braket{\tilde{\psi}}{\tilde{\psi}'} = \sum_{\vb{R}_n, \vb{R}_{n'} }  e^{i \vb{k}\vdot \vb{R}_n} e^{-i \vb{k}'\vdot \vb{R}_{n'}} \alpha(\vb{R}_n-\vb{R}_{n'}) = \sum_{\vb{R}_n} e^{i(\vb{k}-\vb{k}')\vb{R}_n}\sum_{\vb{R}_{n'}} e^{i(\vb{k}-\vb{k}')\vb{R}_{n'}} \alpha(\vb{R}_n-\vb{R}_{n'}) 
\end{align}
\begin{align}\bar{\alpha}(\vb{k}) = \sum_{\vb{R}_{n'}} e^{i(\vb{k}-\vb{k}')\vb{R}_{n'}} \alpha(\vb{R}_n-\vb{R}_{n'}) = 1 + \sum_{\vb{R}_{n'} \neq 0} e^{i\vb{k} \vdot \vb{R}_n} \alpha(\vb{R}_{n'}) = 1 + \tilde{\alpha}(\vb{k})\end{align}
And thus
\begin{align}\braket{\tilde{\psi}}{\tilde{\psi}'} = \delta_{\vb{k}\vb{k}'} \bar{\alpha}(\vb{k})\end{align}
Thus we can normalize and acquire orthonormal states:
\begin{align}\ket{\tilde{\psi}_{norm}} = \frac{1}{\sqrt{1+\tilde{\alpha}(\vb{k})}} \ket{\tilde{\psi}}\end{align}

\begin{align}
\tilde{\epsilon}_{\vb{k}} = \expval{H}{\tilde{\psi}_{\vb{k}}} = \frac{1}{1+\tilde{\alpha}(\vb{k})}  \sum_{n,n'} \int \dd{\vb{r}} e^{i\vb{k} \vdot \qty(\vb{R}_n-\vb{R}_{n'})} \phi^*_{atom} (\vb{r}-\vb{R}_n) \qty[\underbrace{-\frac{-\laplacian}{2m} +V_{atom}(\vb{r})}_{\epsilon_{atom}} + \Delta V(\vb{r}) ] \phi_{atom} (\vb{r}-\vb{R}_n) =\\= \frac{1}{1+\tilde{\alpha}(\vb{k})} \epsilon_{atom} \underbrace{\sum_{n,n'} \int \dd{\vb{r}} e^{i\vb{k} \vdot \qty(\vb{R}_n-\vb{R}_{n'})} \phi^*_{atom} (\vb{r}-\vb{R}_n) \phi_{atom} (\vb{r}-\vb{R}_n)}_{1+\tilde{\alpha}(\vb{k})} +\\+ \frac{1}{1+\tilde{\alpha}(\vb{k})}  \sum_{n,n'} \int \dd{\vb{r}} e^{i\vb{k} \vdot \qty(\vb{R}_n-\vb{R}_{n'})} \phi^*_{atom} (\vb{r}-\vb{R}_n) \Delta V(\vb{r})  \phi_{atom} (\vb{r}-\vb{R}_n) 
\end{align}
We denote
\begin{align}
t(\vb{k}) = -\sum_{n} \int \dd{\vb{r}} e^{i\vb{k} \vdot \vb{R}_n} \phi^*_{atom} (\vb{r}-\vb{R}_n) \Delta V(\vb{r})  \phi_{atom} (\vb{r}-\vb{R}_n) 
\end{align}
and  thus
\begin{align}
\tilde{\epsilon}_{\vb{k}} = \epsilon_{atom}- \frac{t(\vb{k})}{1+\tilde{\alpha}(\vb{k})}  
\end{align}
In a limit of small characteristic atomic length $\frac{l_{atom}}{a} \ll 1$, we get
\begin{align}
\tilde{\alpha}(\vb{k}) \sim \exp(-\qty(\frac{na}{l_{atom}})^2)
\end{align}
i.e.,
\begin{align}
\tilde{\epsilon}_{\vb{k}} \sim \epsilon_{atom}- t(\vb{k})  
\end{align}
For the same reason, due to exponential decay, we can neglect long-distance hopping.



We now take Wannier states of those $\tilde{\psi}_{norm}$:
\begin{align}W(\vb{r} -\vb{R}_n) \simeq \phi_{atom}(\vb{r}-\vb{R}_n) = \sum_{\vb{k}} e^{i\vb{k} \vdot \vb{R}_n} \tilde{\psi}_{norm}(\vb{r})
	\end{align}
We denote $\ket{n} = W_n(\vb{r})$. Those states diagonalize Hamiltonian:
\begin{align}
H = \sum_n\epsilon_{atom}\dyad{n} - \sum_{n,n'} t_{nn'} \dyad{n}{n'} + h.c.
\end{align}

The approximation is broken if, for example two bands are touching each other. Then we can describe two bands together as a single tight-binded band.

\section{Semi-classical dynamics}
First we require $k_BT \sim \hbar \omega \ll \Delta E$, where $\Delta E$ is energy difference between two bands.

If we also require long wavelength $\lambda \gg a$, we get a wavepacket of width $\Delta k$ and center in $\vb{k}_0$:
\begin{align}
W(\vb{r}) = \int \dd{\vb{k}} e^{i\vb{k}\vdot \qty(\vb{r}-\vb{r}_0)}e^{-\frac{(\vb{k}-\vb{k}_0)^2}{\qty(\Delta k)^2}}\sim e^{(\Delta k)^2(\vb{r}-\vb{r}_0)^2}
\end{align}
Thus, using our solution of the Shr\"{o}dinger equation and expanding to the first order (denoting $\vb{k} - \vb{k}_0 = \delta \vb{k}$:
\begin{align}
W(\vb{r}, t) &= \int \dd{\vb{k}} \exp(i\epsilon_{\vb{k}} t + i\vb{k}\vdot \qty(\vb{r}-\vb{r}_0)-\frac{(\vb{k}-\vb{k}_0)^2}{\qty(\Delta k)^2}  ) =\\&=  e^{i\vb{k}_0 \vdot \vb{r}}\int \dd{\vb{k}} \exp(\qty(\epsilon_{\vb{k}_0}  + \pdv{\epsilon}{k} \delta \vb{k})t + i\delta \vb{k} \vdot \qty(\vb{r}-\vb{r}_0)-\frac{(\delta \vb{k})^2}{\qty(\Delta k)^2}  ) =\\&= e^{-(\Delta k)^2 \qty(t\qty(\vb{r} - (\vb{r}_0 + \pdv{\epsilon}{k})))^2} e^{i\vb{k}_0 \vdot \vb{r}}
\end{align}

Group velocity is thus $\grad_{\vb{k}} \epsilon_{\vb{k}}$

\begin{theorem}[Hellmann-Feynman theorem]
	Let $\epsilon_\alpha = \ev{h_\alpha}{\psi_n(\alpha)}$ for some hermitian $h_\alpha$.
	\begin{align}
	\dv{\alpha} \epsilon_\alpha  &=  \mel{\dv{\alpha}\psi_n(\alpha)}{h_\alpha}{\psi_n(\alpha)} + \mel{\psi_n(\alpha)}{\dv{\alpha}h_\alpha}{\psi_n(\alpha)} + \mel{\psi_n(\alpha)}{h_\alpha}{\dv{\alpha}\psi_n(\alpha)} =\\&= \mel{\psi_n(\alpha)}{\dv{\alpha}h_\alpha}{\psi_n(\alpha)} + \epsilon_{\alpha}\qty[\braket{\psi_n(\alpha)}{\dv{\alpha}\psi_n(\alpha)}+\braket{\dv{\alpha}\psi_n(\alpha)}{\psi_n(\alpha)}] =\\&= \mel{\psi_n(\alpha)}{\dv{\alpha}h_\alpha}{\psi_n(\alpha)} + 2\epsilon_{\alpha}\underbrace{\dv{\alpha}\braket{\psi_n(\alpha)}{\psi_n(\alpha)}}_0 = \mel{\psi_n(\alpha)}{\dv{\alpha}h_\alpha}{\psi_n(\alpha)}
	\end{align}
\end{theorem}
\begin{prop}[Rules of semiclassical approximation]
	
	\begin{enumerate}
		\item Wave will move keeping same band index.
		\item Position obeys 
		\begin{align}
		\expval{\vb{r}} \sim \pdv{\epsilon}{\vb{k}}
		\end{align}
		\item 
		\begin{align}
		\hbar \vb{k} = e-\vb{E} - \frac{e}{c} \dot{\vb{r}} \cross \vb{B}
		\end{align}
	\end{enumerate}
\begin{proof}
	Lets write Shr\"{o}dinger equation
	\begin{align}
	\qty(\underbrace{\frac{\hbar \vb{k}^2}{2m}-\frac{2i\hbar \vb{k} \vdot \vb{\grad}}{2m}+\frac{\hbar \laplacian}{2m}}_{H_{\vb{k}}} + V(\vb{r})) \mathcal{U}_{\alpha, \vb{k}}(\vb{r}) = \epsilon_{\alpha, \vb{k}}\mathcal{U}_{\alpha, \vb{k}}(\vb{r}) 
	\end{align}
	Since $\psi_{\alpha, \vb{k}}(\vb{r}) $ is a Bloch wave
	\begin{align}
	\qty(-\frac{\hbar \laplacian}{2m} + V(\vb{r})) \psi_{\alpha, \vb{k}}(\vb{r}) = \epsilon_{\alpha, \vb{k}}\psi_{\alpha, \vb{k}}(\vb{r}) 
	\end{align}
	\begin{align}
	\epsilon_{\alpha, \vb{k}} = \epsilon_{\alpha, \vb{k}_0} \grad_{\vb{k}}\epsilon_{\alpha, \vb{k}_0} \delta \vb{k} + \frac{1}{2} \sum_{i,j} \delta \vb{k}_i \vdot \delta \vb{k}_j \pdv{\epsilon}{\vb{k}_i }{\vb{k}_j}
	\end{align}
	
	\begin{align}
	\epsilon_{\vb{k}} = \ev{H}{\psi_k} = \ev{H_{\vb{k}}}{\mathcal{U}_{\vb{k}}}
	\end{align}
	\begin{align}
	\dv{\vb{k}}\epsilon_{\alpha, \vb{k}} = \dv{\vb{k}}\ev{H_{\vb{k}}}{\mathcal{U}_{\vb{k}}} =  \ev{\dv{\vb{k}}H_{\vb{k}}}{\mathcal{U}_{\vb{k}}} = \ev{\underbrace{e^{i\vb{k}\vdot \vb{r}}\qty(-i\hbar \grad -\vb{k} )e^{-i\vb{k}\vdot \vb{r}}}_m}{\psi_{\vb{k}}}= \ev{\frac{\vb{P}}{m}}{\psi^\alpha_{\vb{k}}} = \vb{v}_{\vb{k}}
	\end{align}
	Thus
	\begin{align}
	\pdv{H_{\vb{k}}}{k} = \frac{-i\hbar \grad -\vb{k}}{m}
	\end{align}
	
	
	
	We also define effective mass tensor:
	\begin{align}
	\frac{1}{\hbar^2}  \pdv{\epsilon}{\vb{k}_i }{\vb{k}_j} = (M^*)_{ij} = 
	\frac{1}{m} \delta_{ij}  - 
	\sum_{\alpha'} \frac{\mel{\psi_{\alpha, \vb{k}}}{P_i}{\psi_{\alpha', \vb{k}}}\mel{\psi_{\alpha', \vb{k}}}{P_i}{\psi_{\alpha, \vb{k}}}}{\epsilon_{\alpha, \vb{k}}-\epsilon_{\alpha', \vb{k}}}
	\end{align}
	
	Now assume we have a particle in a electric field. We ask what is the power dissipated with electric field:
	\begin{align}
	\hbar \pdv{\epsilon_{\vb{k}}}{\vb{k}}  \vdot \dot{\vb{k}} =\pdv{\epsilon_{\vb{k}}}{t} =P_{diss} = - e \vb{E} \vdot \vb{v}_{\vb{k}_0} = -e \vb{E} \vdot \pdv{\epsilon_{\vb{k}}}{\vb{k}}
	\end{align}
	Thus
	\begin{align}
	 \dot{\vb{k}} = -\frac{e}{\hbar} \vb{E} 
	\end{align}
\end{proof}
\end{prop}

\paragraph{Bloch osscilations}
Assume 1D band structure of the form $\epsilon_{\vb{k}} = - 2w\cos(k)$. Since $\dot{\vb{k}} = -\frac{e}{\hbar} \vb{E} $:
\begin{align}
\vb{k} = -\frac{e}{\hbar}\vb{E} t
\end{align}
What will be the velocity?
\begin{align}
\vb{v}(t) = 2W \sin(\frac{e}{\hbar}E t)
\end{align}
Define characteristic $ \tau = \frac{2\pi \hbar}{eE} $:
\begin{align}
\vb{v}(t) = 2W \sin(\frac{2\pi t}{\tau})
\end{align}

This effect was actually observed in cold atoms.
\paragraph{Alternative derivation}
Given PBC in 1D, 
\begin{align}
\psi(x+L) = \psi(x)
\end{align}
and Hamiltonian
\begin{align}
H = \frac{p^2}{2m} + V(x)
\end{align}
We can perceive it as a ring. Adding an Aharonov-Bohm magnetic flux $\Phi$ inside the ring, we create gauge field  (vector potential)
\begin{align}
\vb{A} = \frac{\Phi}{L} \vu{x}
\end{align}
The electromotive force, from Lenz's law
\begin{align}
 -\frac{1}{c}\dot{\Phi}  = \mathcal{E} = E\vdot L
\end{align}
Thus
\begin{align}
\dot{\vb{A}} = - cE \Rightarrow \vb{A} = -cEt
\end{align}
And we rewrite the Hamiltonian
\begin{align}
\vb{p} \to \vb{p} -\frac{e}{c}\vb{A}
\end{align}
\begin{align}
H = \frac{\qty(\vb{p} -\frac{e}{c} \vb{A})^2}{2m} + V(x) = \frac{\qty(\vb{p} -eEt)^2}{2m} + V(x) 
\end{align}
Now we solve the Shr\"{o}dinger equation:
\begin{align}
\tilde{\psi}_k(t) = e^{-\frac{ieAx}{c\hbar}} \psi_k(x)
\end{align}
The boundary condition is not periodic anymore, and substituting flux quantum ($\Phi_0 = \frac{\hbar c}{2\pi e}$) we get:
\begin{align}
\tilde{\psi}_k(L) = e^{-\frac{ieAL}{c\hbar}} \tilde{\psi}_k(0)= e^{-\frac{i\Phi}{c\hbar}} \tilde{\psi}_k(0)= e^{-i 2\pi \qty(\frac{\Phi}{\Phi_0})} \tilde{\psi}_k(0)
\end{align}
From \cref{bloch} 
\begin{align}
\tilde{\psi}_k(x) = e^{ikx} \tilde{\mathcal{U}}_k(x)
\end{align}
\begin{align}
e^{ikL} \tilde{\mathcal{U}}_k(0) = \tilde{\psi}_k(L) =  e^{-\frac{ieAL}{c\hbar}} \tilde{\psi}_k(0) = e^{ik\cdot 0} \tilde{\mathcal{U}}_k(0)e^{-\frac{ieAL}{c\hbar}}
\end{align}
\begin{align}
e^{ikL} = e^{-\frac{ieAL}{c\hbar}}
\end{align}
\begin{align}
e^{i\qty(kL+\frac{eAL}{c\hbar})} = 1 
\end{align}
, i.e., there is a shift in $k$ values
\begin{align}
k = \frac{2\pi n}{L} - \underbrace{\frac{eA}{c\hbar} }_{- \delta k}
\end{align}
We can verify that
\begin{align}
H(A) \psi_k(x) = H(A) e^{\frac{ieAx}{c\hbar}}   \tilde{\psi}_k(t) = e^{\frac{ieAx}{c\hbar}}  H_{A=0} \tilde{\psi}= \epsilon_k e^{\frac{ieAx}{c\hbar}}  \tilde{\psi}_k = \epsilon_k \psi_k  
\end{align}
\begin{align}
H\tilde{\psi}_k = \qty(\frac{p^2}{2m} + V(x)) \tilde{\psi}_k
\end{align}

If we put, for example, half-quantum flux, we'll get $\delta k = \frac{1}{2} \frac{2\pi}{L}$ and thus ground state is degenerate.

If there is electric field, $A=-cEt$,
\begin{align}
\hbar\delta \dot{ \vb{k}} = -e\vb{E}
\end{align} 
If $\vb{E}$ is small enough, such that particles can't change the band, we acquire Bloch oscillation.

\begin{prop}[Semiclassical dynamics in presence of magnetic field]
	\begin{align}
	\hbar \dot{ \vb{k}} = e\vb{E} + \frac{e}{c} \qty(\vb{v}_{\vb{k}} \cross \vb{B})
	\end{align}
\end{prop}

\subsection{Landau--Zener tunneling}
\begin{align}
i\hbar \psi_k^{\alpha'} = \sum_{\alpha} H_{\vb{k}}^{\alpha \alpha'} \psi_{\alpha}
\end{align}
We look at Hamiltonian which depends on time, and only on two bands near the crossing.

We rewrite Hamiltonian as a set of time independent Hamiltonians at $\bar{t}$:
\begin{align}
H_{\bar{t}}^{ad} \psi_{\alpha}^{\bar{t}}(x) = \epsilon_{\alpha}(\bar{t}) \psi_{\alpha'}^{\bar{t}} (x)
\end{align}
\begin{theorem}[Adiabatic theorem]
	A physical system remains in its instantaneous eigenstate if a given perturbation is acting on it slowly enough and if there is a gap between the eigenvalue and the rest of the Hamiltonian's spectrum.
\end{theorem}

\paragraph{Model}

\begin{align}
H = \alpha t \sigma_z + \Delta \sigma_x
\end{align}
The energies
\begin{align}
\epsilon^{ad}(\bar{t}) = \pm \sqrt{(\alpha \bar{t}) + \Delta^2}
\end{align}
Thus the gap is $2\Delta$ and there is  probability to tunnel between bands:
\begin{align}
P_{\mp} (t\to \infty) \sim e^{-\frac{\pi}{2} \frac{\Delta^2}{\alpha}}
\end{align}
which can be found by solving
\begin{align}
\begin{pmatrix}
\dot{ \psi}_\uparrow\\
\dot{ \psi}_\downarrow
\end{pmatrix} = H(t)
\begin{pmatrix}
 \psi_\uparrow\\
 \psi_\downarrow
\end{pmatrix}
\end{align}
Simplest Landau--Zener is
\begin{align}
H = \alpha t \sigma^z + \Delta \sigma^x
\end{align}
We can diagonalize Hamiltonian in each particular moment:
\begin{align}
H^{ad} (t) = \sqrt{(\alpha t)^2+\Delta t^2} \sigma^z
\end{align}
with diagonalizing transformation
\begin{align}
U^{ad} (t) = e^{\frac{1}{2} \theta(t) \sigma^y}
\end{align}
where $\theta(t) = \arctan(\frac{\alpha t}{\Delta})$.
The evolution operator is
\begin{align}
U^\infty &= \mathcal{T} e^{-i \int_{-\infty}^{\infty} \dd{t'} H(t')} = \prod_{t' = -\infty}^\infty  e^{-i H(t') \dd{t'}} = \prod_{t' = -\infty}^\infty  U^{ad}(t') {U^{ad}}^\dagger(t') e^{-i H(t') \dd{t'}} U^{ad}(t) { U^{ad}}^\dagger(t') =\\&= \prod_{t' =-\infty}^\infty  U^{ad}(t') e^{-i \Delta E(t') \dd{t'} \sigma^z} { U^{ad}}^\dagger(t')
\end{align}
But 
\begin{align}
{ U^{ad}}^\dagger(t'+\dd{t})U^{ad}(t') = \qty({ U^{ad}}^\dagger(t') + \dv{U^\dagger}{t'}  \dd{t}) U^{ad}(t') \approx 1 + \frac{i}{2} \sigma^y \dot{\theta} \dd{t}= 1 + \frac{i}{2} \sigma^y  \qty(\frac{\alpha \Delta}{(\alpha t)^2+\Delta^2})\dd{t}
\end{align}
Substituting into the product and taking first order
\begin{align}
\mel{\uparrow}{U^\infty}{\downarrow} &= \sum_{t'} \exp(-i \int_{t'}^\infty \dd{t''} \sqrt{(\alpha t'')^2+\Delta t^2}) \frac{i}{2}  \qty(\frac{\alpha \Delta}{(\alpha t')^2+\Delta^2}) \exp(-i \int_{-\infty}^{t'} \dd{t''} \sqrt{(\alpha t'')^2+\Delta t^2}) + \order{\alpha^2} =\\&=
\exp(-i \int_{-\infty}^\infty \dd{t''} \sqrt{(\alpha t'')^2+\Delta t^2})  \int \dd{t'} \frac{i}{2}  \qty(\frac{\alpha \Delta}{(\alpha t')^2+\Delta^2}) \exp(-2i \int_{-\infty}^{t'} \dd{t''} \sqrt{(\alpha t'')^2+\Delta t^2}) + \order{\alpha^2}
\end{align}

We can use saddle point approximation:
\begin{align}
\int_{-\infty}^c \dd{t} e^{i S(t)} p(t) \simeq e^{i S(\bar{t})} p(\bar{t}) \int \dd{\delta t} \exp( -\pdv[2]{S}{t} (\delta t) + \order{(\delta t)^2} )
\end{align}
where $\bar{t}$ is such that
\begin{align}
\eval{\pdv{S}{t}}_{t=\bar{t}}=0
\end{align}
However, the solution is complex. We can substitute it since our function is entire on complex plane and continue our function to the saddle point. (In $p$ we substitute $\Re{\tau} = 0$ since else it diverges):
\begin{align}
\mel{\uparrow}{U^\infty}{\downarrow} &\simeq  
\exp(-i \int_{-\infty}^\infty \dd{t''} \sqrt{(\alpha t'')^2+\Delta t^2}) \frac{\alpha}{\Delta} \exp(-2\int_0^{\frac{\Delta}{\alpha}} \dd{\tau'} \sqrt{\Delta^2 - (\alpha \tau')^2}) \sim\\&\sim \exp(-2 \qty(\frac{\Delta}{\alpha})^2 \int_0^1 \dd{u} \sqrt{1-u^2}) = e^{-\frac{\pi}{4} \qty(\frac{\Delta}{\alpha})^2}
\end{align}
\paragraph{Electric field}
\begin{align}
\vb{E} = -\frac{1}{c}\dot{\vb{A}}  - \grad{\phi}
\end{align}
If $\dot{\vb{A}} =0 $ and $\phi(\vb{x}) = -\vb{x}\vdot \vb{E}$, we have
\begin{align}
H_E = \frac{p^2}{2m} + e\vb{E} \vdot \vb{x} + V(x)
\end{align} 
But this Hamiltonian diverges in $\pm \infty$. Instead we can choose $\Lambda = c\vb{E} \vdot \vb{x}t$ for gauge transform
\begin{align}
\vb{A}' = \vb{A} + \grad{\Lambda(\vb{x},t)}
\end{align} 
acquiring 
\begin{align}
\vb{A}' = -c\vb{E} t
\end{align}
\section{Boltzmann equation}
In equilibrium
\begin{align}
f^{(0)}_{\vb{k}, \alpha} = \frac{1}{\exp(\frac{\epsilon_{\vb{k}}^\alpha-\mu}{T})+1} = n_{\vb{k}}
\end{align}

If we look at phase space, then our states are incompressible liquid in a phase space, i.e., total number of particles doesn't change:
\begin{align}
\dv{f_{\vb{k}}}{t}  = \pdv{f_{\vb{k}}}{t} + \pdv{f_{\vb{k}}}{\vb{k}} \vdot \dot{\vb{k}} + \pdv{f_{\vb{k}}}{\vb{r}} \vdot \dot{\vb{r}} = 0
\end{align}
However, in addition, in can get scattered:
\begin{align}
P_{\vb{k}, \vb{k}'} = 2\pi \abs{\mel{\psi_{k'}}{V}{\psi_{k}}}^2 \delta(\epsilon_{\vb{k}} - \epsilon_{\vb{k}'}-\omega)
\end{align}
Thus
\begin{align}
\dv{f_{\vb{k}}}{t}  = \pdv{f_{\vb{k}}}{t} + \pdv{f_{\vb{k}}}{\vb{k}} \vdot \dot{\vb{k}} + \pdv{f_{\vb{k}}}{\vb{r}} \vdot \dot{\vb{r}}  = -\qty(\pdv{f_{\vb{k}}}{t})_{\text{collisions}}
\end{align}

If we had $f_{\vb{k}} \qty(\vb{k}, \vb{r}, t)$, we could calculate observables, such as charge density, current, heat current, magnetization:
\begin{align}
\rho(\vb{r}, t) &= \sum_{\vb{k}} f_{\vb{k}} (\vb{r},t)\\
\vb{j}(\vb{r}, t) &= e\sum_{\vb{k}} f_{\vb{k}} (\vb{r},t) \vdot \vb{v}_{\vb{k}}\\
\vb{j}_Q(\vb{r}, t) &= \sum_{\vb{k}} f_{\vb{k}} (\vb{r},t)(E_{\vb{k}} - e\mu) \vdot \vb{v}_{\vb{k}}\\
m_z(\vb{r}, t) &= \sum_{\vb{k},s} f_{\vb{k}} (\vb{r},t) \qty(\frac{\hbar}{2} \cdot s)\\
\end{align}

Lets evaluate $\qty(\pdv{f_{\vb{k}}}{t})_{\text{collisions}}$:
\begin{align}
-\qty(\pdv{f_{\vb{k}}}{t})_{\text{collisions}} = \sum_{\vb{k'}} P_{\vb{k}' \leftarrow \vb{k}} f_{\vb{k}} (1- f_{\vb{k}'}) - P_{\vb{k} \leftarrow \vb{k}'} f_{\vb{k}'} (1- f_{\vb{k}}) 
\end{align}
In equilibrium
\begin{align}
f=f^0 &= \frac{1}{\exp(\beta(\epsilon_{\vb{k}} - \mu)) -1 }\\
1-f=1-f^0 &= \frac{\exp(\beta(\epsilon_{\vb{k}} - \mu))}{\exp(\beta(\epsilon_{\vb{k}} - \mu)) -1 }= \exp(\beta(\epsilon_{\vb{k}} - \mu)) f^0\\
\end{align}
Then\begin{align}
-\qty(\pdv{f_{\vb{k}}}{t})_{\text{collisions}} = \sum_{\vb{k'}} P_{\vb{k}'\leftarrow \vb{k}} \exp(\beta(\epsilon_{\vb{k}} - \mu)) f^0_{\vb{k}}f^0_{\vb{k}'} - P_{\vb{k} \leftarrow \vb{k}'} \exp(\beta(\epsilon_{\vb{k}'} - \mu)) f^0_{\vb{k}}f^0_{\vb{k}'} 
\end{align}

But
\begin{align}
\frac{P_{\vb{k}'\leftarrow \vb{k}}}{P_{\vb{k} \leftarrow \vb{k}'}} = e^{-\beta(\epsilon_{\vb{k}'} - \epsilon_{\vb{k}})}
\end{align}
i.e., everything vanishes. But $P_{\vb{k}'\leftarrow \vb{k}}$ depends on thing breaking translational symmetry:  impurities, phonons, electron-electron interactions.

\paragraph{Relaxation time approximation}
For phonons we can approximate with
\begin{align}
-\qty(\pdv{f_{\vb{k}}}{t})_{\text{collisions}} = \frac{f(\vb{k},\vb{r},t) - f^0(\vb{k})}{\tau}
\end{align}
where $\tau$ is relaxation time. The approximation makes sense since we are interested in first order approximation.
\paragraph{Impurities approximation}
\begin{align}
P_{\vb{k}', \vb{k}} &= 2\pi \abs{\mel{\vb{k}}{V^{im}}{\vb{k}'}}^2 \delta(\epsilon_k - \epsilon_{k'})\\
-\qty(\pdv{f_{\vb{k}}}{t})_{\text{collisions}} &= \sum_{\vb{k}'} \qty\big[f_{\vb{k}}(1-f_{\vb{k}'}) -f_{\vb{k}'}(1-f_{\vb{k}}) ]\abs{V^{im}_{\vb{k},\vb{k}'}}^2 \delta\qty(\epsilon_{\vb{k}}-\epsilon_{\vb{k}'})
\end{align}
where
\begin{align}
V^{im} &= \sum_i \vartheta(\vb{r}-\vb{R}_i)\\
V_{\vb{k}, \vb{k}'} &= \int \psi^* V^{im} \psi \dd{x} = \frac{1}{V} \sum_i e^{i\qty(\vb{k}-\vb{k}')\vb{R}_i}\vartheta
\end{align}
Then
\begin{align}
\expval{\abs{V^{im}_{\vb{k},\vb{k}'}}^2} = \expval{\sum_{i,j} e^{i(\vb{k}-\vb{k'})(\vb{r}_i\vb{R}_j)}\abs{V_{\vb{k}, \vb{k}'} }^2 } = \sum_{i} \delta_{ij} \abs{V_{\vb{k}, \vb{k}'} }^2 = N \vartheta_{\vb{k}, \vb{k}'}  
\end{align}
where $N$ is number of impurities.
\begin{align}
-\qty(\pdv{f_{\vb{k}}}{t})_{\text{collisions}} &= \sum_{\vb{k}'} \qty\big[f_{\vb{k}}(1-f_{\vb{k}'}) -f_{\vb{k}'}(1-f_{\vb{k}}) ]\abs{V^{im}_{\vb{k},\vb{k}'}}^2 \delta\qty(\epsilon_{\vb{k}}-\epsilon_{\vb{k}'})
\end{align}

Thus
\begin{align}
 \pdv{f_{\vb{k}}}{t} + \pdv{f_{\vb{k}}}{\vb{k}} \vdot \dot{\vb{k}} + \pdv{f_{\vb{k}}}{\vb{r}} \vdot \dot{\vb{r}}  = -\qty(\pdv{f_{\vb{k}}}{t})_{\text{collisions}} = -\frac{f-f^0}{\tau}
\end{align}

We can approximate
\begin{align}
f = f^0 + \delta f
\end{align}

Since
\begin{align}
\hbar \dot{\vb{k}} = e\vb{E}
\end{align}
we can make scattering approximation
\begin{align}
\qty(\pdv{f}{t}) = -\frac{\delta f}{\tau}
\end{align}
where $\tau$ depends on scattering Hamiltonian: impurities (elastic scattering), phonons (inelastic), electron-electron interactions.

\subsection{Presence of electric field}
Let
\begin{align}
\vb{E}_{\vb{\omega}} = \Re{ e^{i\omega t}\vb{E} }
\end{align}
we assume homogeneous field, i.e. $\pdv{f}{\vb{r}}=0$:
\begin{align}
\pdv{\delta f_{\vb{k}}}{t}  - \frac{e}{\hbar} \vb{E} e^{i\omega t} \qty(\pdv{f_{\vb{k}}}{\vb{k}}) = \frac{\delta f_{\vb{k}}}{\tau}
\end{align}
Since we are interested in first order part, we can replace $ \qty(\pdv{f_{\vb{k}}}{\vb{k}})$ with $ \qty(\pdv{f^0}{\vb{k}}) = \pdv{f^0}{E} \cdot \pdv{E}{\vb{k}}$:
The solution is 
\begin{align}
\delta f(t) = e^{i \omega t} \delta f_{\vb{k}, \omega}
\end{align}
\begin{align}
e \vb{v}_{\vb{k}} \vdot \vb{E} e^{i\omega t} \qty(-\pdv{f^0}{E}) = \delta f_{\vb{k}, \omega} e^{i\omega t} \qty(\frac{1}{\tau} - i \omega)
\end{align}
i.e.,
\begin{align}
\delta f_{\vb{k}, \omega} = \frac{e \vb{v}_{\vb{k}} \vdot \vb{E} \tau}{1-i\omega t}  \qty(-\pdv{f^0}{E}) 
\end{align}
Then
\begin{align}
J_x(\omega) &= \sigma_{xx}(\omega) E^x_\omega\\
J^x_{k,\omega} &= e \sum_{\vb{k} \in BZ} \delta f_{\vb{k}, \omega} v_k^x\\
J^x_\omega &= \underbrace{e^2 \sum_{\vb{k} \in BZ} \qty(-\pdv{f^0}{E})_{\vb{k}} \frac{{ v_k^x }^2 \tau}{1-i\omega \tau}}_{\sigma_{xx}(\omega)} \cdot E^x_\omega\\
\end{align}
Note we can approximate pretty well 
\begin{align}
\qty(-\pdv{f^0}{E}) \stackrel{T\to 0}{\sim} \delta(\epsilon-\mu)
\end{align}

\subsection{Parabolic bands} We can recove Drude model in a parabolic band (i.e., around extrema). 
\begin{align}
\epsilon_{\vb{k}} &= \frac{\hbar^2 \vb{k}^2 }{2m^*}\\
\sum_{\vb{k}} \delta (\epsilon_{\vb{k}} {\vb{k}^*}^2 &= \frac{1}{3} \sum_{\vb{k}} \delta (\epsilon_{\vb{k}} \abs{\vb{v}}^2
\end{align}
\begin{align}
\vb{v}^2= \qty(v_1^2)^2+\qty(v_2^2)^2+\qty(v_3^2)^2 = \frac{2}{3} \sum_{\vb{k}} \frac{\delta(\epsilon_{\vb{k}}-\mu)\epsilon_{\vb{k}}}{m^*}
\end{align}
\begin{align}
\epsilon_{\vb{k}} = \frac{\hbar^2 \vb{k}^2}{2m^*} = \frac{{m^*}^2 \vb{v}_{\vb{k}}^2}{2m^*} = \frac{{m^*} \vb{v}_{\vb{k}}^2}{2}
\end{align}
\begin{align}
1 = \int \dd{\epsilon } \sum_{\vb{k}} \delta(\epsilon - \epsilon_{\vb{k}}) = \frac{2}{3} \int \dd{\epsilon} \frac{\rho(\epsilon) \delta(\epsilon-\mu) \epsilon }{m^*}= \frac{2}{3m^*} \rho(\epsilon_F)\epsilon_F
\end{align}
\begin{align}
\rho(\epsilon_F) = \pdv{N(\epsilon_F)}{\epsilon_F}
\end{align}
when
\begin{align}
N(\epsilon_F) &= \qty(\frac{1}{2\pi})^3 \qty(\frac{4\pi}{3}) k_F^3\\
k_F &=  \qty(\frac{2m \epsilon_F}{\hbar^2})^{\frac{1}{2}}\\
N(\epsilon_F)  &=  \frac{1}{2\pi^2}\qty(\frac{2m \epsilon_F}{\hbar^2})^{\frac{3}{2}}\\
\rho(\epsilon_F) &= \frac{1}{2\pi^2}\qty(\frac{2m }{\hbar^2})^{\frac{3}{2}}\frac{3}{2}\epsilon_F^{\frac{1}{2}}
\end{align}
Then
\begin{align}
\sigma = e^2 \tau \qty[\frac{2}{3m^*} \frac{1}{2\pi^2} \qty(\frac{2m^*}{\hbar^2})^{\frac{3}{2}} \epsilon_F^{\frac{3}{2}} \qty(\frac{3}{2}) ] = \frac{e^2 \tau}{m^*} \qty(\frac{1}{2\pi^2}) k_F^3
\end{align}
\end{document}
