We can approximate
\begin{align}
f = f^0 + \delta f
\end{align}

Since
\begin{align}
\hbar \dot{\vb{k}} = e\vb{E}
\end{align}
we can make scattering approximation
\begin{align}
\qty(\pdv{f}{t}) = -\frac{\delta f}{\tau}
\end{align}
where $\tau$ depends on scattering Hamiltonian: impurities (elastic scattering), phonons (inelastic), electron-electron interactions.

\subsection{Presence of electric field}
Let
\begin{align}
\vb{E}_{\vb{\omega}} = \Re{ e^{i\omega t}\vb{E} }
\end{align}
we assume homogeneous field, i.e. $\pdv{f}{\vb{r}}=0$:
\begin{align}
\pdv{\delta f_{\vb{k}}}{t}  - \frac{e}{\hbar} \vb{E} e^{i\omega t} \qty(\pdv{f_{\vb{k}}}{\vb{k}}) = \frac{\delta f_{\vb{k}}}{\tau}
\end{align}
Since we are interested in first order part, we can replace $ \qty(\pdv{f_{\vb{k}}}{\vb{k}})$ with $ \qty(\pdv{f^0}{\vb{k}}) = \pdv{f^0}{E} \cdot \pdv{E}{\vb{k}}$:
The solution is 
\begin{align}
\delta f(t) = e^{i \omega t} \delta f_{\vb{k}, \omega}
\end{align}
\begin{align}
e \vb{v}_{\vb{k}} \vdot \vb{E} e^{i\omega t} \qty(-\pdv{f^0}{E}) = \delta f_{\vb{k}, \omega} e^{i\omega t} \qty(\frac{1}{\tau} - i \omega)
\end{align}
i.e.,
\begin{align}
\delta f_{\vb{k}, \omega} = \frac{e \vb{v}_{\vb{k}} \vdot \vb{E} \tau}{1-i\omega t}  \qty(-\pdv{f^0}{E}) 
\end{align}
Then
\begin{align}
J_x(\omega) &= \sigma_{xx}(\omega) E^x_\omega\\
J^x_{k,\omega} &= e \sum_{\vb{k} \in BZ} \delta f_{\vb{k}, \omega} v_k^x\\
J^x_\omega &= \underbrace{e^2 \sum_{\vb{k} \in BZ} \qty(-\pdv{f^0}{E})_{\vb{k}} \frac{{ v_k^x }^2 \tau}{1-i\omega \tau}}_{\sigma_{xx}(\omega)} \cdot E^x_\omega\\
\end{align}
Note we can approximate pretty well 
\begin{align}
\qty(-\pdv{f^0}{E}) \stackrel{T\to 0}{\sim} \delta(\epsilon-\mu)
\end{align}

\subsection{Parabolic bands} We can recove Drude model in a parabolic band (i.e., around extrema). 
\begin{align}
\epsilon_{\vb{k}} &= \frac{\hbar^2 \vb{k}^2 }{2m^*}\\
\sum_{\vb{k}} \delta (\epsilon_{\vb{k}} {\vb{k}^*}^2 &= \frac{1}{3} \sum_{\vb{k}} \delta (\epsilon_{\vb{k}} \abs{\vb{v}}^2
\end{align}
\begin{align}
\vb{v}^2= \qty(v_1^2)^2+\qty(v_2^2)^2+\qty(v_3^2)^2 = \frac{2}{3} \sum_{\vb{k}} \frac{\delta(\epsilon_{\vb{k}}-\mu)\epsilon_{\vb{k}}}{m^*}
\end{align}
\begin{align}
\epsilon_{\vb{k}} = \frac{\hbar^2 \vb{k}^2}{2m^*} = \frac{{m^*}^2 \vb{v}_{\vb{k}}^2}{2m^*} = \frac{{m^*} \vb{v}_{\vb{k}}^2}{2}
\end{align}
\begin{align}
1 = \int \dd{\epsilon } \sum_{\vb{k}} \delta(\epsilon - \epsilon_{\vb{k}}) = \frac{2}{3} \int \dd{\epsilon} \frac{\rho(\epsilon) \delta(\epsilon-\mu) \epsilon }{m^*}= \frac{2}{3m^*} \rho(\epsilon_F)\epsilon_F
\end{align}
\begin{align}
\rho(\epsilon_F) = \pdv{N(\epsilon_F)}{\epsilon_F}
\end{align}
when
\begin{align}
N(\epsilon_F) &= \qty(\frac{1}{2\pi})^3 \qty(\frac{4\pi}{3}) k_F^3\\
k_F &=  \qty(\frac{2m \epsilon_F}{\hbar^2})^{\frac{1}{2}}\\
N(\epsilon_F)  &=  \frac{1}{2\pi^2}\qty(\frac{2m \epsilon_F}{\hbar^2})^{\frac{3}{2}}\\
\rho(\epsilon_F) &= \frac{1}{2\pi^2}\qty(\frac{2m }{\hbar^2})^{\frac{3}{2}}\frac{3}{2}\epsilon_F^{\frac{1}{2}}
\end{align}
Then
\begin{align}
\sigma = e^2 \tau \qty[\frac{2}{3m^*} \frac{1}{2\pi^2} \qty(\frac{2m^*}{\hbar^2})^{\frac{3}{2}} \epsilon_F^{\frac{3}{2}} \qty(\frac{3}{2}) ] = \frac{e^2 \tau}{m^*} \qty(\frac{1}{2\pi^2}) k_F^3
\end{align}